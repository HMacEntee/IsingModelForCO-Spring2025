\documentclass{article}
\usepackage[utf8]{inputenc}
\usepackage[a4paper,margin=.75in,footskip=0.25in]{geometry}
\usepackage{setspace} 
\usepackage{braket}
\usepackage{amssymb}

\begin{document}

\begin{enumerate}
\item \textbf{Project Title:} Solving Combinatorial Problem with the Ising Model and Machine Learning

\item \textbf{Investigator List:} Levy Lin, Holden Mac, Entee, Yanglet Xiaoyang Liu

\item \textbf{Project Description}
	\begin{enumerate}
		\item \textbf{Motivation}
		\newline
		The Ising model is a powerful model for combinatorial problems, capable of mapping various complex problem states for the discovery of their low energy Hamiltonian or optimal solution. Through this model many NP-Hard problems such as the MaxCut, Traveling Salesman, and Graph-Coloring problem can be solved for at large-scale, opposed to the classical landscape. When combined with the advantages of machine learning, the efficiency and performance of the combinatorial applications of the Ising model can be improved. This projects seeks to explore and discover improvements to pre-exisitng and proposed Ising models with applications towards solving combinatorial problems with machine learning. 

		\item \textbf{Key Methodology}

		\item \textbf{Expected Outcomes}

		
	\end{enumerate}

\item \textbf{Estimated Project Timeline}
\end{enumerate}

\textbf{References}
\begin{thebibliography}{00}
\bibitem{b1}Carlson, C., Davies, E., Kolla, A., \& Perkins, W. (2022). Computational thresholds for the fixed-magnetization Ising model. Proceedings of the 54th Annual ACM SIGACT Symposium on Theory of Computing (STOC 2022), 1459–1472. https://doi.org/10.1145/3519935.3520003
\bibitem{b2}Lucas, A. (2014). Ising formulations of many NP problems. Frontiers in Physics, 2(5). https://doi.org/10.3389/fphy.2014.00005
\bibitem{b3}Benlic, U., \& Hao, J.-K. (2013). Breakout Local Search for the Max-Cut problem. Engineering Applications of Artificial Intelligence, 26(3), 1162–1173. https://doi.org/10.1016/j.engappai.2012.09.001
\bibitem{b4}Chen, C., Chen, R., Li, T., Ao, R., \& Wen, Z. (2023). Monte Carlo policy gradient method for binary optimization. arXiv. https://arxiv.org/abs/2307.00783


\end{thebibliography}
\end{document}
