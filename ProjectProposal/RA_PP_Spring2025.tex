\documentclass{article}
\usepackage[utf8]{inputenc}
\usepackage[a4paper,margin=.75in,footskip=0.25in]{geometry}
\usepackage{setspace} 
\usepackage{braket}
\usepackage{amssymb}

\begin{document}

\begin{enumerate}
\item \textbf{Project Title:} Training Transformer Network to Find Ground State of Ising Models

\item \textbf{Investigator List:} Levy Lin, Holden Mac Entee, Yanglet Xiaoyang Liu

\item \textbf{Project Description}
\begin{enumerate}
	\item \textbf{Motivation}
	\newline
	The Ising model is a powerful model for combinatorial problems, capable of mapping various complex problem states for the discovery of their low energy Hamiltonian or optimal solution. Through this model many NP-Hard problems such as the MaxCut, Traveling Salesman, and Graph-Coloring problem can be solved for at large-scale, opposed to the classical landscape. When combined with the advantages of machine learning, the efficiency and performance of the combinatorial applications of the Ising model can be improved. This projects seeks to explore and discover improvements to pre-exisitng and proposed Ising models with applications towards solving combinatorial problems with machine learning. 

	\item \textbf{Key Methodology}
		
	\begin{enumerate}
			
		\item \textbf{\underline{Task 1: Read 3 $\sim$ 5 references and find the associated datasts of Ising models.}} 
		\item \textbf{\underline{Task 2: Reproduce Professor Liu's codes and pipeline for baseline.}} 

		\item \textbf{\underline{Task 3: Implement a variant of transformer network and use reinforcement learning algorithm to train it under Professor Liu's guidance.}} 

	\end{enumerate}

	\item \textbf{Expected Outcomes}
	
	\begin{enumerate}
		\item Beat existing solvers by 0.5\% on graphs ranging from 300 to 500 nodes.
		\item Scale our algorithm to large graph instances ranging from 1,000 to 3,000 nodes and outperform existing solvers by 2-3\%.
	\end{enumerate}

\end{enumerate}

\item \textbf{Estimated Project Timeline}
\begin{enumerate}
	\item 12/1/2024 $\sim$ 12/31/2024: Read papers; identify datasets to benchmark; set up solver and pipeline
	\item 1/1/2025 $\sim$ 2/28/2025: Outperform existing solvers on medium and large datasets
	\item 3/1/2025 $\sim$ 4/30/2025: Write paper to submit to NeurIPS 2025
\end{enumerate}

\end{enumerate}

\begin{thebibliography}{00}
\bibitem{b1}Carlson, C., Davies, E., Kolla, A., \& Perkins, W. (2022). Computational thresholds for the fixed-magnetization Ising model. Proceedings of the 54th Annual ACM SIGACT Symposium on Theory of Computing (STOC 2022), 1459–1472. https://doi.org/10.1145/3519935.3520003
\bibitem{b2}Lucas, A. (2014). Ising formulations of many NP problems. Frontiers in Physics, 2(5). https://doi.org/10.3389/fphy.2014.00005
\bibitem{b3}Benlic, U., \& Hao, J.-K. (2013). Breakout Local Search for the Max-Cut problem. Engineering Applications of Artificial Intelligence, 26(3), 1162–1173. https://doi.org/10.1016/j.engappai.2012.09.001
\bibitem{b4}Chen, C., Chen, R., Li, T., Ao, R., \& Wen, Z. (2023). Monte Carlo policy gradient method for binary optimization. arXiv. https://arxiv.org/abs/2307.00783


\end{thebibliography}
\end{document}
