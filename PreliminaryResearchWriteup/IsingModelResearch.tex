\documentclass[conference]{IEEEtran}
\IEEEoverridecommandlockouts
% The preceding line is only needed to identify funding in the first footnote. If that is unneeded, please comment it out.
%Template version as of 6/27/2024

\usepackage{cite}
\usepackage{amsmath,amssymb,amsfonts}
\usepackage{algorithmic}
\usepackage{graphicx}
\usepackage{textcomp}
\usepackage{xcolor}
\usepackage{verbatim}
\usepackage{braket}
\usepackage{tikz}
\usepackage{tablefootnote}
\usetikzlibrary{quantikz}
\def\BibTeX{{\rm B\kern-.05em{\sc i\kern-.025em b}\kern-.08em
    T\kern-.1667em\lower.7ex\hbox{E}\kern-.125emX}}
\begin{document}

\title{Ising Model Solver for Combinatorial Optimization Problem}

\author{\IEEEauthorblockN{Levy Lin}
\IEEEauthorblockA{\textit{Department of Computer Science} \\
\textit{Department of Economics} \\
\textit{Rensselaer Polytechnic Institute}\\
Troy, United States \\
linl9@rpi.edu}
\and
\IEEEauthorblockN{Holden Mac Entee}
\IEEEauthorblockA{\textit{Department of Computer Science} \\
\textit{Department of Electrical, Computer \& Systems Engineering} \\
\textit{Rensselaer Polytechnic Institute}\\
Troy, United States \\
macenh@rpi.edu}
}

\maketitle

\section{Introduction}
The Ising model is a mathematical model of a ferromagnetic material under the presence of a magnetic field, which allows for the analysis of the thermodynamics of a system of ferromagnetic particles\cite{b1}. When applied with the spin glass magnetic state, the Ising spin-glass model can be utilized to analyze the thermodynamics of unordered or chaotic systems. Particularly, the spin-glass Ising model aligns each ferromagnetic particle in a lattice with a random spin-interactions depicted as: $\sigma \in \{-1, +1\}$. For the purpose of this document, we will consider the spin-glass Ising model to compute NP-Hard problems.

Formally, the $N$-spin Ising problem aims to find the configuration of spins that minimizes the energy Hamiltonian, 
\begin{equation}
H = -\sum_{i,j<N}J_{ij}\sigma_i\sigma_j - \sum_ih_i\sigma_i
\end{equation}
where $J_{ij}$ represents the coupling coefficient, negligible for non-neighboring spins, and $h_i$ represents the magnetic field acting on spin $i$. This document will explore the solving of the NP-Hard problem MaxCut using the Ising spin glass model. As such, the energy of MaxCut can be modeled by the following equation:
\begin{equation}
Cut(s) = \frac{1}{2}\sum_{(i,j)\in E}(1-\sigma_i\sigma_j)
\end{equation}
To map this to the Hamiltonian for Ising spin glass and maintain the objective of the problem, we aim to solve for the ground state of $H(s) = -Cut(s)$. In other words, we have the following equation representing the Hamiltonian we will strive to minimize:
\begin{equation}
H(s) = \sum_{(i,j)\in E}\sigma_i\sigma_j
\end{equation}
Due to the difficulty of finding the minimum Hamiltonian of a chaotic systems, the Ising spin glass model is a NP-Hard problem for classical computers. Naturally, we are able to correlate this property to all NP-Hard problems, and can justifiably state that Ising spin glasses can be polynomially mapped to all other NP-Hard problems\cite{b2}.

\section{Existing Methods}
We will be comparing our results to algorithms such as BLS, CPLEX, Gurobi, and MCPG. A table is shown below documenting the results the BLS and MCPG algorithms achieved on Gset datasets according to their referenced papers. Whilst the results the Gurobi and CPLEX algorithms were achieved on Syn datasets. Both algorithms used the same problem formulation of an Binary Integer approximation\cite{b5} and was implemented using the PuLP python library. 

\begin{table}[htbp]
\caption{Gset Dataset Results}
\begin{center}
\begin{tabular}{|c|c|c|c|c|c|c|}
\hline
\textbf{Graph} & \textbf{Nodes} & \textbf{Edges} & \textbf{MCPG} & \textbf{BLS} & \textbf{Gurobi} \\
\hline 
G14 & 800& 4,694& 3,064 &3,064 & 3,017\\
\hline
G15 & 800& 4,661& 3,050 & 3,050 & 2,990\\
\hline
G22 & 2,000& 19,990& 13,359 &13,359 & 12,938\\
\hline
G49 & 3,000& 6,000& 6,000 &6,000 & 6,000\\
\hline
G50 & 3,000& 6,000& 5,880 &5,880 & 5880\\
\hline
G55 & 5,000& 12,468& 10,294 &10,294 & 10,010\\
\hline
G70 & 10,000& 9,999 & 9,595 &9,541 & 9,570\\
\hline
\end{tabular}
\label{tab1}
\end{center}
\end{table}
\begin{table}[htbp]
\caption{Syn Dataset Results}
\begin{center}
\begin{tabular}{|c|c|c|c|c|c|c|}
\hline
\textbf{Graph} & \textbf{Nodes} & \textbf{Edges} & \textbf{Gurobi} & \textbf{CPLEX} \\
\hline 
P\_20\_ID0 & 20 & 63 & 46 & 46 \\
\hline
P\_40\_ID0 & 40 & 144& 109 & 109\\
\hline
P\_100\_ID0 & 100 & 384 & 282 & 282 \\
\hline
P\_200\_ID0 & 200 & 784 & 581 & * \\
\hline
P\_300\_ID0 & 300 & 1182 & 868 & * \\
\hline
P\_400\_ID0 & 400 & 1582 & 1157 & * \\
\hline
P\_500\_ID0 & 500 & 1982 & 868 & * \\
\hline
\end{tabular}
\label{tab1}
\end{center}
\end{table}

The most impressive algorithm shown is the Monte Carlo Policy Gradient (MCPG) method. This algorithm has specific features that contribute to its success, including: a filter function that acts to reduce the probability the algorithm will fall into local minima; a sampling procedure with filter function that starts from the best solution found previously and aims to maintain diversity; a modified policy gradient algorithm to update the probabilistic model; and a probabilistic model that guides the sampling procedure towards potentially good solutions.





\begin{thebibliography}{00}
\bibitem{b1}Carlson, C., Davies, E., Kolla, A., \& Perkins, W. (2022). Computational thresholds for the fixed-magnetization Ising model. Proceedings of the 54th Annual ACM SIGACT Symposium on Theory of Computing (STOC 2022), 1459–1472. https://doi.org/10.1145/3519935.3520003
\bibitem{b2}Lucas, A. (2014). Ising formulations of many NP problems. Frontiers in Physics, 2(5). https://doi.org/10.3389/fphy.2014.00005
\bibitem{b3}Benlic, U., \& Hao, J.-K. (2013). Breakout Local Search for the Max-Cut problem. Engineering Applications of Artificial Intelligence, 26(3), 1162–1173. https://doi.org/10.1016/j.engappai.2012.09.001
\bibitem{b4}Chen, C., Chen, R., Li, T., Ao, R., \& Wen, Z. (2023). Monte Carlo policy gradient method for binary optimization. arXiv. https://arxiv.org/abs/2307.00783
\bibitem{b5} https://www.tcs.tifr.res.in/~prahladh/teaching/2009-10/limits/lectures/lec03.pdf

\end{thebibliography}

\end{document}







%--------------FORMATTING ASSISTANCE-------------------









\begin{comment}


\begin{IEEEkeywords}
component, formatting, style, styling, insert.
\end{IEEEkeywords}

\begin{equation}
a+b=\gamma\label{eq}
\end{equation}



\subsection{Figures and Tables}\label{FAT}
\paragraph{Positioning Figures and Tables} Place figures and tables at the top and 
bottom of columns. Avoid placing them in the middle of columns. Large 
figures and tables may span across both columns. Figure captions should be 
below the figures; table heads should appear above the tables. Insert 
figures and tables after they are cited in the text. Use the abbreviation 
``Fig.~\ref{fig}'', even at the beginning of a sentence.

\begin{table}[htbp]
\caption{Table Type Styles}
\begin{center}
\begin{tabular}{|c|c|c|c|}
\hline
\textbf{Table}&\multicolumn{3}{|c|}{\textbf{Table Column Head}} \\
\cline{2-4} 
\textbf{Head} & \textbf{\textit{Table column subhead}}& \textbf{\textit{Subhead}}& \textbf{\textit{Subhead}} \\
\hline
copy& More table copy$^{\mathrm{a}}$& &  \\
\hline
\multicolumn{4}{l}{$^{\mathrm{a}}$Sample of a Table footnote.}
\end{tabular}
\label{tab1}
\end{center}
\end{table}

\begin{figure}[htbp]
\centerline{\includegraphics{fig1.png}}
\caption{Example of a figure caption.}
\label{fig}
\end{figure}


\end{comment}
