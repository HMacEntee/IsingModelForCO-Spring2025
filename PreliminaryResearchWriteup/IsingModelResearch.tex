\documentclass[conference]{IEEEtran}
\IEEEoverridecommandlockouts
% The preceding line is only needed to identify funding in the first footnote. If that is unneeded, please comment it out.
%Template version as of 6/27/2024

\usepackage{cite}
\usepackage{amsmath,amssymb,amsfonts}
\usepackage{algorithmic}
\usepackage{graphicx}
\usepackage{textcomp}
\usepackage{xcolor}
\usepackage{verbatim}
\usepackage{braket}
\usepackage{tikz}
\usetikzlibrary{quantikz}
\def\BibTeX{{\rm B\kern-.05em{\sc i\kern-.025em b}\kern-.08em
    T\kern-.1667em\lower.7ex\hbox{E}\kern-.125emX}}
\begin{document}

\title{Ising Model Solver for Combinatorial Optimization Problem}

\author{\IEEEauthorblockN{Levy Lin}
\IEEEauthorblockA{\textit{Department of Computer Science} \\
\textit{Department of Economics} \\
\textit{Rensselaer Polytechnic Institute}\\
Troy, United States \\
linl9@rpi.edu}
\and
\IEEEauthorblockN{Holden Mac Entee}
\IEEEauthorblockA{\textit{Department of Computer Science} \\
\textit{Department of Electrical, Computer \& Systems Engineering} \\
\textit{Rensselaer Polytechnic Institute}\\
Troy, United States \\
macenh@rpi.edu}
}

\maketitle

\begin{comment}
\begin{abstract}
The integration of quantum computing in finance holds significant promise in showcasing the power of quantum supremacy. By utilizing the quantum approximate optimization algorithm (QAOA), we are able to perform the MaxCut operation on a connected graph of stocks to perform portfolio optimization. In QAOA, a quantum-classical loop consisting of a parameterized quantum circuit and classical optimizer solves a combinatorial problem. This poster shows how we were able to map stock market information to a graph that can undergo the operation of QAOA. Our implementation focuses on using Modern Portfolio Theory (MPT) to evaluate stocks, and aims to separate the portfolio to diversify the portfolio, minimizing risk and maximizing return.

We demonstrate the functionality of our implementation on a weighted 5-complete graph. This graph considers 5 stocks and minimizes the correlation between the two to create a diverse portfolio. We achieved a fidelity of 81\% using 8 repetitions on the simulated environment.
\end{abstract}
\end{comment}

The Ising model, a mathematical model of a magnetic material, provides a description of the energy of a system of atoms\cite{b1}. It is modeled by random spin-interactions depicted as: $\sigma \in \{-1, +1\}$. For the purpose of this document, we will consider the spin-glass Ising model. That is, the spins are randomly distributed between $\pm1$. Ising spin glass models are NP-Hard problems for classical computers. Naturally, we are able to correlate this property to all NP-Hard problems, and can be justifiably stated that Ising spin glasses are able to be polynomially mapped to all other NP-Hard problems\cite{b2}.

Formally, the $N$-spin Ising problem aims to find the configuration of spins that minimizes the energy Hamiltonian, 
\begin{equation}
H = -\sum_{i,j<N}J_{ij}\sigma_i\sigma_j - \sum_ih_i\sigma_i
\end{equation}
where $J_{ij}$ represents the coupling coeffcient, negligible for non-neighboring spins. This document will explore the solving of the NP-Hard problem MaxCut using the Ising spin glass model. As such, the energy of MaxCut can be modeled by the following equation:
\begin{equation}
Cut(s) = \frac{1}{2}\sum_{(i,j)\in E}(1-\sigma_i\sigma_j)
\end{equation}
To map this to the Hamiltonian for Ising spin glass and maintain the objective of the problem, we aim to solve for the ground state of $H(s) = -Cut(s)$. In other words, we have the following equation representing the Hamiltonian we will strive to minimize:
\begin{equation}
H(s) = \sum_{(i,j)\in E}\sigma_i\sigma_j
\end{equation}





\begin{thebibliography}{00}
\bibitem{b1} Carlson, C., Davies, E., Kolla, A., \& Perkins, W. (2022). Computational thresholds for the fixed-magnetization Ising model. Proceedings of the 54th Annual ACM SIGACT Symposium on Theory of Computing (STOC 2022), 1459–1472. https://doi.org/10.1145/3519935.3520003
\bibitem{b2} Lucas, A. (2014). Ising formulations of many NP problems. Frontiers in Physics, 2(5). https://doi.org/10.3389/fphy.2014.00005
\bibitem{b3} King, A. D., Bernoudy, W., King, J., Berkley, A. J., \& Lanting, T. (2018). Emulating the coherent Ising machine with a mean-field algorithm. arXiv. https://doi.org/10.48550/arXiv.1806.08422



\end{thebibliography}

\end{document}







%--------------FORMATTING ASSISTANCE-------------------









\begin{comment}


\begin{IEEEkeywords}
component, formatting, style, styling, insert.
\end{IEEEkeywords}

\begin{equation}
a+b=\gamma\label{eq}
\end{equation}



\subsection{Figures and Tables}\label{FAT}
\paragraph{Positioning Figures and Tables} Place figures and tables at the top and 
bottom of columns. Avoid placing them in the middle of columns. Large 
figures and tables may span across both columns. Figure captions should be 
below the figures; table heads should appear above the tables. Insert 
figures and tables after they are cited in the text. Use the abbreviation 
``Fig.~\ref{fig}'', even at the beginning of a sentence.

\begin{table}[htbp]
\caption{Table Type Styles}
\begin{center}
\begin{tabular}{|c|c|c|c|}
\hline
\textbf{Table}&\multicolumn{3}{|c|}{\textbf{Table Column Head}} \\
\cline{2-4} 
\textbf{Head} & \textbf{\textit{Table column subhead}}& \textbf{\textit{Subhead}}& \textbf{\textit{Subhead}} \\
\hline
copy& More table copy$^{\mathrm{a}}$& &  \\
\hline
\multicolumn{4}{l}{$^{\mathrm{a}}$Sample of a Table footnote.}
\end{tabular}
\label{tab1}
\end{center}
\end{table}

\begin{figure}[htbp]
\centerline{\includegraphics{fig1.png}}
\caption{Example of a figure caption.}
\label{fig}
\end{figure}


\end{comment}
